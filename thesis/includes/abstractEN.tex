The main subject of this master thesis is face classification using convolutional neural networks on operating system Android.
The thesis begins with section dedicated to general analysis of biometric systems and the most popular technique for face recognition.
Base on this analysis we concluded, that the base technic for face recognition is nowadays using neural networks.
We describe neural networks in second section, where we talk about base principles of neurons, we define mathematical model of neuron and later explain process of neural networks training.
We end the second section by explanation of convolutional neural networks, providing informations about the main stages convolutional neural network is based on - convolution and pooling, and later by describing the common problems which occurs during training of neural networks.
In practical part of this thesis we talk about Facenet - model of convolutional neural networks trained for extraction of 128-dimensional vector of face embeddings.
Facenet is acquiring very impressive results in classification based on its extracted embeddings, mostly thanks to using triplet loss function.
As for classification algorithm we decided for linear SVM, which resulted in successful classification varying from $ 98,8\% $ to $ 99,4\% $ depending on number of training classes.
Trained models are later applied in Android applications, which is using face detector based on Histogram of Oriented Gradients.
The extracted face chips are later inferred through model of Facenet implemented in Tensorflow library, resulting in embeddings, based on which we classify person on an image.