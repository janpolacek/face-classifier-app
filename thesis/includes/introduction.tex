\indent Témou tejto diplomovej práce je použitie konvolučných neurónových sietí na zariadení Android OS na klasifikáciu tvárí.
Android OS je populárnym operačným systémom mobilných zariadení, s podielom na trhu $ 87,7\% $, za čo vďačí najmä veľkému množstvo výrobcov,atraktívnej cene a taktiež dostupnosti veľkého množstva aplikácii.
Konvolučné neurónové siete sa na druhej strane tešia veľkej popularite najmä v oblasti výskumu spracovania obrazu, vyžadujúci vysoký výpočetný výkon a taktiež veľké množstvo anotovaných trénovacích dát.
Táto diplomová práca sa teda venuje spojeniu týchto dvoch aspektov do jednej aplikácie pre Android OS, schopnej detegovať polohu tváre a taktiež ju správne priradiť správnej identite.
Tento cieľ sme rozdelili do 4 kapitol, v ktorých postupne objasňujeme celú problamatiku spojenú s týmto zadaním.
V každej kapitole sa snažíme uviesť čitateľa do problematiky, pričom v jej podsekciách vysvetľujeme všetky dôležité pojmy, prípadne ich výkad dopĺňame názornou grafickou ukážkou.
V prvej kapitole sa venujeme problematike biometrických systémov a najmä na metódy a techniky vedúce k rozpoznávaniu tváre.
Druhou kapitolou popisujeme základy neurónových sietí, ich vznik a podobnosť s ľudským neurónom, jeho matematický model, trénovanie neurónových sietí a problémy spojené s trénovaním.
Postupným výkladom sa dostaneme až samotným konvolučným neurónovým sieťam.
Treťou kapitolou popisujeme proces trénovania klasifikátora tvárí.
Oboznámime sa z niektorými známymi databázami tvárí, možnosťami detekcie tváre a popíšeme model konvolučnej neurónovej siete slúžiaci k extrahovaniu príznakov tváre a natrénovaniu samotného klasifikátora.
V poslednej kapitole sa venujeme návrhu aplikácie, popisu použitia natrénovaného klasifikátora na zariadení.
Na záver zhrnieme všetky výsledky a poznatky do kompaktného celku a navrhneme možné vylepšenia do budúcnosti.



