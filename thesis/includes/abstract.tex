Hlavným cieľom ktorému sa táto práce venuje je klasifikácia tvárí pomocou konvolučnej siete na operačnom systéme Android.
Práce začína kapitolou obsahujúcou všeobecnú analýzu biometrických systémov a ich rozdeleniu a popísaniu populárnych a používaných techník pri rozpoznávaní tváre.
Touto analýzou sme dospeli k predpokladu, že neurónové siete sú v súčastnosti najvhodnejšou technikou pre rozpoznávanie tvárí.
Neurónové siete ako také popisujeme v druhej kapitole, počnúc od definície matematického modelu neurónu, procesu trénovania neurónových sietí.
Kapitolu zakončujeme definíciou konvolučných neurónových sietí, vysvetleniu základných operácií v konvolučných neurónových sieťach - konvolúcia a pooling a identifikácii problémov spojených s trénovaním neurónových sietí.
V praktickej časti tejto práce popisujeme model konvolučnej neurónovej siete - Facenet, zameraný na extrakciu príznakov tváre do 128-dimenzionálneho kompaktného vektora.
Vďaka výberu vhodných trénovacích dát a použitiu chybovej funkcie trojíc dosahuje tento model výborné výsledky pri klasifikácii extrahovaných príznakov.
Za klasifikátor sme si zvolili SVM s linárnym jadrom, s ktorým sme dosiahli výsledky pohybujúce sa od $ 98,8\% $ po $ 99,4\% $ v závislosti od počtu identít.
Natrénované modely následne aplikujeme v aplikácii Android, v ktorej využívame detektor tváre založený na Histrogram of Oriented Gradients.
Detekované tváre následne podliehajú procesu extrakcie príznakov v modely Facenetu implementovaného v knižnici Tensorflow, ktoré sú následne klasifikované použitím natrénovaného SVM klasifikátora.
