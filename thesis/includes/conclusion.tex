\noindent Problematikou ktorou sa zaoberá táto práca bolo naštudovanie problematiky klasifikácie ľudskej tváre. 
V úvodnej kapitole\ref{l:techn} sme sa venovali problematike biometrických systémov ako takých, popísali sme základné rozdelenie biometrických systémov a venovali sa najmä technikám slúžiacim k rozpoznávaniu tváre.
Zaujímavým ja javí najmä presun od techník využívajúce štatistickú analýzu dát, ako sú napríklad PCA, LDA či LFA smerom k neurónovým sieťam.

\noindent Neurónovým sieťam sme sa detailnejšie venovali v kapitole \ref{l:nns}, kde sme prebrali základný koncept a matematický model neurónov a neurónových sietí,aktivačné funkcie, trénovanie neurónovej siete či najčastejšie problémy spojené s trénovaním modelov.
Nakoniec kapitoly sme sa zamerali na špecifickú kategóriu neurónových sietí - konvolučným neurónovým siete, a prebrali v čom sa odlišujú od klasických neurónových sietí,z akých častí pozostáva a v čom predchádza klasické neurónové siete.
O ich výborných vlastnostiach svedčí aj množstvo súčasných projektov založených práve na CNN, ako sú napríklad projekty Thingscoop\cite{agermani43} ktorého cieľom je klasifikácia a filtrovanie videí v závislosti od objektov ktoré sa v ňom nachádzajú alebo \cite{Kim14f} slúžiaci ku klasifikácii viet.

\noindent V kapitole \ref{l:fcnt} tejto práce sme sa venovali návrhu konvolučnej neurónovej siete, slúžiacej ku klasifikácií tvárí.
Využili sme pritom existujúcí návrh modelu schopného extrahovať príznaky tváre do kompaktného 128-dimenzionálneho vektora - Facenet\cite{schroff2015facenet}.
Popísali sme princípy na ktorý je Facenet postavený - najmä chybovú funkciu trojíc, ktorá mala enormný podiel na dosiahnutí výborných výsledkov pri klasifikácií $ 99,63\% $) úspešnosť pri využití vektora príznakov, ktoré Facenet extrahuje z tváre, čím markantne prekonal doterajšie riešenia\\
\noindent V ďalších sekciách kapitoly sme sa venovali procesu trénovania modelu Facenetu.
Preskúmali sme dostupné datasety tvárí, pričom práve správny výber datasetu sa podľa niektorých štúdií\cite{Centerfo}, zdá byť často dôležitejší než samotný algoritmus použitý v CNN.
Nemalú časť kaptily sme venovali možnostiam detekcie tváre, a porovnali sme dva najperspektívnejšie  riešenia - detekciu založenú na HOG v knižnici Dlib a detekciu tvárí pomocou kaskádnych konvolučných neurónových sietí - MTCNN.
Zistili sme, že MTCNN\cite{mtcnn} je oveľa vhodnejšou možnosťou, so schopnosťou detegovať oveľa zložitejšie kompozície snímkov tvárí, avšak za cenu vyšších nárokov na výpočtovú silu. \\
\noindent Pokračovali natrénovaním SVM modelu klasifikátora na základe extrahovaných príznakov z Facenetu, pričom sme porovnali úspešnosť klasifikácie pri variabilnom počte tried osôb, ktoré sme mali rozpoznať.

\noindent Finálnym krokom\ref{l:andapp} tejto práce bolo implementovať mobilnú aplikáciu na zariadení Android OS.
Mobilné zariadenia ani v súčasnosti nedosahujú výkon porovnateľný s výpočtovou silou počítačov disponujúcich silnou grafickou kartou.
Z toho dôvodu sme sa venovali v časti\ref{l:opt} venovali možnostiam vedúcim k zníženiu výpočtovej náročnosti modelov.
Touto optimalizáciou sa nám podarilo zlepšiť výkon Facenetu o $ 15\% $ a zmenšiť jeho veľkosť o $ 75\% $.
Cieľom tejto práce bolo vytvoriť aplikáciu schopnú klasifikácie tváre v čo najlepšom čase.
Z tohoto dôvodu sme k detekcií tváre z kamery využili C++ implementáciu detektora z knižnice Dlib, a vytvorili interface, cez ktorý je možné s knižnicou komunikovať z Java kódu.
Pre detekované tváre, sme podobne ako pri časti trénovania klasifikátora, extrahovali príznaky, ktoré sme následne klasifikovali natrénovaným modelom SVM.


\noindent Výsledkom tejto práce je aplikácia Face classifier, ktorá deteguje tváre 120ms, extrahuje príznaky za 500ms a klasifikuje za 1ms, a teda dostaneme 10fps detekciu polohy tváre a 2fps identifikáciu osoby a natrénované modely klasifikátora.
Rovnako sme pripravili užívateľskú príručku popisujúcu všetky kroky potrebné k inicializácií projektu a inštalácii aplikácie, problémy s ktorými sme sa stretli - najmä v oblasti kompilácie knižníc pre Android či rôzne bash skripty potrebné k trénovaniu a optimalizácii klasifikátora.

\noindent Touto prácou sme splnili všetky body zadania práce.
Napriek tomu sme však pri skúmaní tejto problematiky objavili niekoľko možných zlepšení.
Ide najmä o preskúmanie možností tvorby optimálnejších CNN pre mobilné zariadenia ako je SqueezeNet\cite{IandolaMAHDK16} a MobileNet\cite{HowardZCKWWAA17}, ktoré vyžadujú oveľa menší výpočtový výkon za cenu len malého zhoršenia presnosti.
Možnosť ďalšieho zlepšenia vidíme v optimalizácie metódy detekcie na zariadení Android, implementujúc napríklad v ROLO\cite{redmon2016you},a teda metóde založenej kombinácií detekcie objektu a jeho následným trackovaním v ďalších snímkoch, namiesto použitia detekcie pre každý snímok.

